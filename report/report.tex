\documentclass[12pt]{article}
\usepackage[en]{prettytex/base}
\usepackage{prettytex/math}
\usepackage{prettytex/code}
\usepackage{prettytex/contract}
\usepackage[toc, acronym, style=long3col, indexonlyfirst=true, nogroupskip=true]{glossaries}
\usepackage[backend=biber, citestyle=authoryear, bibstyle=authoryear, hyperref=true, sorting=none, maxbibnames=99]{biblatex}
\usepackage{csquotes}
\usepackage[nameinlink]{cleveref}
\usepackage{dsfont}

% Reporting
% from: https://static.uni-graz.at/fileadmin/projekte/biotechmedgraz/Programme/Lab_Rotation/Lab_Rotation_Program_Guidelines_2023.pdf
% The LRP awardee will be required to submit a short final report about his/her scientific activities during
% the rotation. The report must be signed by the lab rotation mentor.
% Report details: Max. 6 pages, composed of:
% • Summary
% • Introduction
% • Methods
% • Results
% • Future perspectives
% Reports must be submitted at the end of the fourth month of the lab rotation. The approval of the
% report is the basis for the transferal of the second instalment of the stipend.

% Was ist beim Endbericht zu beachten?
% from: https://biotechmedgraz.at/de/mitgliederbereich/programme/fragen-und-antworten-zum-biotechmed-graz-lab-rotation-programm/
% Der von Ihrem/Ihrer Betreuer*in unterzeichnete Endbericht ist spätestens am letzten Tag Ihrer Lab Rotation per E-Mail an die BioTechMed-Graz Geschäftsstelle zu senden. Bitte beachten Sie Details zum Endbericht in den Lab Rotation Richtlinien.

% https://physics.stackexchange.com/questions/302553/deriving-biot-savart-law-from-maxwells-equations

\title{Current Density Imaging via MRI}
\author{Peter Julius Waldert}
\date{29\th of February, 2024}

\makenoidxglossaries
\newacronym{mri}{MRI}{Magnetic Resonance Imaging}
\newacronym{mrt}{MRT}{Magnetic Resonance Tomography}
\newacronym{fft}{FFT}{Fast Fourier Transform}
\newacronym{ifft}{IFFT}{Inverse Fast Fourier Transform}
\newacronym{cdi}{CDI}{Current Density Imaging}

\addbibresource{../literature/sources.bib}

\begin{document}
  \makeatletter
  \begin{center}
    {\Huge \@title} \\
    {\large BioTechMed-Graz Lab Rotation Report}
    \vspace{.5cm}

    of {\large \@author}
    \vspace{.3cm}

    supervised by \textbf{Prof. Kristian Bredies} at the \\
    Institute of Mathematics and Scientific Computing (IMSC), \\
    University of Graz, Austria
    \vspace{.3cm}

    {\@date}.
  \end{center}
  \makeatother

  \section{Summary}
  This lab rotation project was concerned with the development, exploration and implementation of a new optimisation approach for the reconstruction of current density / conductivity of tissue within a \gls{mri} setup.

  The project was split into two steps: First, simulating the magnetic field modulation given a current density field inside a \textit{phantom} (the \textit{forward} procedure).
  And second, reconstructing current density (including conductivity) using a novel optimisation model, based on measurements of the z-component of the magnetic field, with insight gained from the forward procedure (which, in turn is referred to as the \textit{backward} procedure).

  \begin{figure}[H]
    \centering
    \includegraphics[width=\linewidth]{../figures/komaui.png}
    \caption{Koma UI}
    \label{fig:komaui}
  \end{figure}

  \section{Introduction}
  The overall intention of \gls{mri} is to turn a source of contrast into an image for clinicians to use, mainly for the identification of different (potentially malignant) tissue types.
  In the usual setting, this source of contrast is one of $T_1$, $T_2$ or $T_2^*$, three material constants.
  Visualising these material properties in an image therefore allows to differentiate between different types of tissue visually.

  \subsection{Mathematical Introduction to Magnetic Resonance Imaging}
  The general equation governing the behaviour of these \textit{spin} objects is the Bloch equation:
  \begin{equation}
    \frac{\dd \vec{M}}{\ddt} = \gamma (\vec{M} \times \vec{B})\,.
  \end{equation}

  \subsection{The Biot-Savart Law}
  For a given current density $\vec{j}(t): \Omega \to \R^3$ \footnote{
    The current density $\vec{j} = nq\vec{v}$, describing the flow of $n$ charges $q$ with velocity $\vec{v}$, may be related to \textit{current} $I$ through the infinitesimal $\vec{j} \,\dd^3\vec{x} = I \,\dd\vec{x}$.
  } on a domain $\Omega \subseteq \R^3$ at time $t \in \R^+$, Maxwell's fourth equation in differential form
  \begin{equation*}
    \nabla \times \vec{B} = \mu_0 \left(\vec{j} + \epsilon_0 \frac{\partial \vec{E}}{\partial t}\right)\,,
    \label{eq:maxwell-4}
  \end{equation*}
  relates the curl of the magnetic field $\vec{B}(t): \Omega \to \R^3$ to the current density $\vec{j}(t)$ and the temporal rate of change in the \textit{electric} field $\vec{E}(t): \Omega \to \R^3$.
  $\epsilon_0$ and $\mu_0$ are the electric permittivity and magnetic permeability of free space, respectively.
  The current density $\vec{j}$ is also connected to the electric field $\vec{E}$ through the, also position-dependent, electrical conductivity $\sigma: \Omega \to \R^+$
  $$\vec{j} = \sigma \vec{E}\,.$$

  In the \textbf{electrostatic case} (when the electric field $\vec{E}$ is indepent of time $t$), the last term in the above equation will vanish and we arrive at
  \begin{equation}
    \nabla \times \vec{B} = \mu_0 \vec{j}\,.
    \label{eq:maxwell-4-electrostatic}
  \end{equation}
  This equation, together with the freedom of divergence of the magnetic field (Maxwell's second equation), $\nabla \cdot \vec{B} = 0$, leads to the \textbf{Biot-Savart law}
  \begin{equation}
    \vec{B}(\vec{x})
    = \frac{\mu_0}{4\pi} \int_\Omega \frac{\vec{j}(\vec{y}) \times (\vec{x} - \vec{y})}{\norm{\vec{x} - \vec{y}}_2^3} \,\dd\vec{y}
    = \frac{\mu_0}{4\pi} \int_\Omega \frac{\vec{j}(\vec{y}) \times \vec{y}}{\norm{\vec{y}}_2^3} \,\dd\vec{y}
    \label{eq:biot-savart}
  \end{equation}
  which provides an explicit expression for the magnetic field contribution $\vec{B}(\vec{x})$ at position $\vec{x} \in \Omega$ given a current density field $\vec{j}$.

  \subsection{Current Density Imaging}

  \section{Methods}
  For visualisation and as a general framework, we built upon the \textbf{KomaMRI} toolchain and Julia software package \parencite{2022-koma-mri}.
  This choice made sense because of its feature-richness and further extensibility.

  \subsection{Gridded Data Format}
  In order to use KomaMRI together with the \gls{fft}-based, we had to adapt the storage format.
  % used .h5

  \subsection{FFT-accelerated evaluation of the Biot-Savart law}
  Starting from the Biot-Savart law introduced above (cf. \Cref{eq:biot-savart}), we observe that splitting the cross-product ($\times$) into its three respective components
  \begin{align*}
    B_1(\vec{x}) & = \frac{\mu_0}{4\pi} \int_{\Omega} \frac{j_2(\vec{y}) \cdot (\vec{x}_3 - \vec{y}_3) - j_3(\vec{y}) \cdot (\vec{x}_2 - \vec{y}_2)}{\norm{\vec{x} - \vec{y}}_2^3} \,\dd\vec{y} \,, \\
    B_2(\vec{x}) & = \frac{\mu_0}{4\pi} \int_{\Omega} \frac{j_3(\vec{y}) \cdot (\vec{x}_1 - \vec{y}_1) - j_1(\vec{y}) \cdot (\vec{x}_3 - \vec{y}_3)}{\norm{\vec{x} - \vec{y}}_2^3} \,\dd\vec{y} \,, \\
    B_3(\vec{x}) & = \frac{\mu_0}{4\pi} \int_{\Omega} \frac{j_1(\vec{y}) \cdot (\vec{x}_2 - \vec{y}_2) - j_2(\vec{y}) \cdot (\vec{x}_1 - \vec{y}_1)}{\norm{\vec{x} - \vec{y}}_2^3} \,\dd\vec{y} \,,
  \end{align*}
  according to the explicit representation of the cross-product $\times$ in three spatial dimensions, most notably allows us to express $B_1$, $B_2$ and $B_3$ as convolution integrals
  $$B_1(\vec{x}) = \frac{\mu_0}{4\pi} \int_{\Omega} \big[j_2(\vec{y}) g_3(\vec{x} - \vec{y}) - j_3(\vec{y}) g_2(\vec{x} - \vec{y})\big]\,\dd\vec{y} = \frac{\mu_0}{4\pi} \big[(j_2 *_{\Omega} g_3) - (j_3 *_{\Omega} g_2)\big](\vec{x})\,,$$
  with $g_1, g_2, g_3: \Omega \to \R$, $g_1(\vec{x}) = \frac{\vec{x}_1}{\norm{\vec{x}}_2^3}$, $g_2(\vec{x}) = \frac{\vec{x}_2}{\norm{\vec{x}}_2^3}$ and $g_3(\vec{x}) = \frac{\vec{x}_3}{\norm{\vec{x}}_2^3}$ and the respective analogs for $B_2$ and $B_3$ \parencite{2020-biot-savart-evaluation-fft}.

  Using the convolution theorem for two functions $f$ and $g$
  $$\mathcal{F}[f * g](\vec{k}) = \mathcal{F}[f](\vec{k}) \cdot \mathcal{F}[g](\vec{k}) \quad\text{where}\quad \mathcal{F}[f](\vec{k}) := \int f(\vec{x}) \e^{-\i \vec{k} \vec{x}}\,\dd\vec{x}\,,$$
  (for more information, we refer the reader to \cite{2022-convolution-theorem}), the respective convolution integrals may be evaluated ``much faster'' using the \gls{fft} and \glstext{ifft}.
  More precisely, this speedup is due to the computational complexity cost reduction from $\mathcal{O}(n^2)$ to $\mathcal{O}(n \log n)$. % TODO

  The first component of the magnetic field $B_1 = \{\vec{B}\}_1$ may then be expressed as
  $$B_1(\vec{x}) = \frac{\mu_0}{4\pi}\mathcal{F}^{-1}\left[\mathcal{F}(j_2) \cdot \mathcal{F}(g_3) - \mathcal{F}(j_3) \cdot \mathcal{F}(g_2)\right](\vec{x})\,,$$
  and analogous expressions may be found for the second and third component $B_2$ and $B_3$, respectively.
  Importantly, we can explicitly evaluate
  $$\mathcal{F}[g_n](\vec{k}) = \mathcal{F}\left[\vec{x} \mapsto \frac{x_n}{\norm{\vec{x}}_2^3}\right](\vec{k}) = -4\pi\i \frac{k_n}{\norm{\vec{k}}_2^2}\,, \quad\text{for}\; n=1,2,3\,,$$
  where $\i$ is the imaginary unit,
  which allows for a direct evaluation of the above using only the \gls{fft}, \glstext{ifft}, addition and multiplication \parencite{2020-biot-savart-evaluation-fft}.

  In code, this may be implemented like so:
  \begin{minted}{julia}
cross(a1, a2, a3, b1, b2, b3) = (
  a2 .* b3 - a3 .* b2,
  -(a1 .* b3 - a3 .* b1),
  a1 .* b2 - a2 .* b1
)

function calculate_magnetic_field(cdp::CurrentDensityPhantom)::VectorField
  # [obtain M_range, calculate g1, g2, g3 on the entire frequency domain]
  c1, c2, c3 = cross(fft(pad_jx), fft(pad_jy), fft(pad_jz), g1, g2, g3)
  B1, B2, B3 = real(ifft(c1)), real(ifft(c2)), real(ifft(c3))
  return mu_0 .* (B1[M_range...], B2[M_range...], B3[M_range...])
end
  \end{minted}

  \begin{figure}[H]
    \centering
    \begin{subfigure}[t]{0.48\textwidth}
      \centering
      \includegraphics[width=\textwidth]{../figures/demo-cdp-j-field.pdf}
      \caption{Exemplary uniform, homogeneous current density vector field $\vec{j}(\vec{x}) = {j}_0 \hat{\vec{e}}_z \mathds{1}_{\vec{x} \in \Omega_s}$ on a sample rectangular domain.}
      \label{fig:demo-cdp-j-field}
    \end{subfigure}
    \hfill
    \begin{subfigure}[t]{0.48\textwidth}
      \centering
      \includegraphics[width=\textwidth]{../figures/demo-cdp-b-field.pdf}
      \caption{Magnetic field $\vec{B}(\vec{x})$ resulting from the homogeneous current density field depicted in \Cref{fig:demo-cdp-j-field}.}
      \label{fig:demo-cdp-b-field}
    \end{subfigure}
  \end{figure}

  \subsection{The Optimisation Approach}
  In order to be able to reconstruct the current density vector field $\vec{j}$ and conductivity scalar field $\sigma$, we employed the following optimisation model:
  \begin{align*}
    \vec{B}^*, \sigma^* & = \argmin_{\vec{B}, \sigma} &  & \frac{1}{2} \norm{\{\vec{B}\}_3 - B_3^0}_2^2 + \frac{\alpha}{2} \int_\Omega \frac{\norm{\nabla \times B(\vec{x})}_2}{\sigma(\vec{x})} \,\dd\vec{x} + R(\sigma) \\
                        & \text{subject to}           &  & \nabla \cdot \vec{B} = \div \vec{B} = 0                                                                                                                        \\
                        & \text{and}                  &  & \sigma \in [\sigma_0, \sigma_1]
  \end{align*}

  Using Maxwell's second equation, one can obtain the corresponding current density $\vec{j}^*$ from Maxwell's fourth equation in the electrostatic case (\Cref{eq:maxwell-4-electrostatic}),
  $$\vec{j}^* = \frac{1}{\mu_0} (\nabla \times \vec{B}^*)$$

  The optimisation was performed using the Optim.jl optimisation library \parencite{2018-optim-jl}.

  \section{Results}
  \subsection{Usage as a Julia Package}

  \section{Future Perspectives}
  \subsection{Alternate Formulation and Splitting Method}
  As was suggested by Prof. Bredies, one could approach the problem using an alternative model and optimisation routine:
  \begin{align*}
    \vec{B}^*, \sigma^* & = \argmin_{\vec{B}, \sigma}
  \end{align*}

  \vspace{4cm}
  \SignatureAndDate{(Peter Waldert)}

  \vspace{2cm}
  \SignatureAndDate{(Prof. Kristian Bredies)}

  \pagebreak
  \printbibliography
  \printnoidxglossary[type=acronym, title={Acronyms}]
\end{document}
